\documentclass{uc3mpracticas}

\usepackage{helvet}
\usepackage{caption}
\usepackage{listings}
\renewcommand{\familydefault}{\sfdefault}


%%%%%%%%%%%%%%%%%%%%%%%%%%%%%%%%%%%%%%%%%%%%%%%%%%%%%%%%%%%%%%%%%%%%%%%%%%%%%%%%
%%%                   Plantilla Prácticas UC3M                               %%%
%%%                Universidad Carlos III de Madrid                          %%%
%%%                   Alejandro Valverde Mahou                               %%%
%%%%%%%%%%%%%%%%%%%%%%%%%%%%%%%%%%%%%%%%%%%%%%%%%%%%%%%%%%%%%%%%%%%%%%%%%%%%%%%%

%Permitir cabeceras y pie de páginas personalizados
\pagestyle{fancy}

%Path por defecto de las imágenes
\graphicspath{ {./images/} }

%Declarar formato de encabezado y pie de página de las páginas del documento
\fancypagestyle{doc}{
  %Cabecera
  \headerpr[1]{Planificación Automática}{}{Ingeniería del Conocimiento}
  %Pie de Página
  \footerpr{}{\textbf{UC3M}}{{\thepage} de \pageref{LastPage}}
}

%Declarar formato de encabezado y pie del título e indice
\fancypagestyle{titu}{%
  %Cabecera
  \headerpr{}{}{}
  %Pie de Página
  \footerpr{}{}{}
}


\appto\frontmatter{\pagestyle{titu}}
\appto\mainmatter{\pagestyle{doc}}


\begin{document}
  %Comienzo formato título
  \frontmatter


  %Portada 1 (Centrado todo)
  \centeredtitle{Images/LogoUC3M.png}{Grado en Ingeniería Informática}{Curso 2020/2021}{Ingeniería del Conocimiento}{Práctica 2: Planificación Automática}

  \vspace{55mm}

  \authors{Alba Reinders Sánchez}{100383444}{Alejandro Valverde Mahou}{100383383}{}{}{Grupo 83}{Leganés}

  \newpage

  %Índice
  \tableofcontents

  \newpage

  %Comienzo formato documento general
  \mainmatter

  \section{Introducción}


  \section{Manual técnico}

  A continuación se describe el dominio creado explicando las acciones y los predicados.

  \subsection{Predicados}


  \begin{itemize}
    \item \textbf{Paciente ?p}: indica el nombre del paciente que va a recibir la sesión.
    \item \textbf{Identificado ?idp}: indica que el paciente ha sido identificado por el robot.
    \item \textbf{Saludado ?sp}: indica que el paciente ha sido saludado por el robot.
    \item \textbf{JuegoExplicado ?jp}: indica que se le ha explicado el juego al paciente.
    \item \textbf{NumRondas ?nr}: indica el número de rondas máximas de la sesión.
    \item \textbf{RondaActual ?r}: indica el número de la ronda actual de la sesión.
    \item \textbf{Turno ?t}: indica de quién es el turno.
    \item \textbf{CartaEnRonda ?c ?r}: indica en qué ronda se va a usar cada carta.
    \item \textbf{}
  \end{itemize}


  \subsection{Acciones}

  \begin{itemize}
    \item \textbf{Identificar Paciente}: acción del robot que permite pasar del estado inicial al momento en el que se ha reconocido al paciente.

      \textit{Precondiciones}: que exista un paciente, y que no haya sido identificado todavía.

      \textit{Efectos}: el paciente pasa a estar identificado.

    \item \textbf{Saludar Paciente}: acción del robot por la que se saluda a un paciente que ya ha sido identificado, antes de empezar un juego.

      \textit{Precondiciones}: que exista un paciente que haya sido identificado, pero no saludado.

      \textit{Efectos}: saludar al paciente

    \item \textbf{Explicar Juego}: acción del robot en la que este explica las reglas del juego de las cartas.

      \textit{Precondiciones}: que exista un paciente que haya saludado, pero que no se le haya explicado el juego.

      \textit{Efectos}: se le explica el juego al paciente.

    \item \textbf{Comenzar Sesión}: acción del robot que da comienzo al juego.

      \textit{Precondiciones}: que haya un paciente al que se le haya explicado el juego y que exista un número de rondas definidas.

      \textit{Efectos}: se crea la primera ronda.

    \item \textbf{Comenzar Ronda}: acción del robot que da comienzo a una nueva ronda.

      \textit{Precondiciones}: que el juego esté creado y que se haya preparado una ronda.

      \textit{Efectos}: es el turno del paciente.

  \end{itemize}









  \section{Manual de usuario}


  \section{Pruebas realizadas}


  \section{Conclusiones}


  \section{Comentarios personales}




\end{document}
