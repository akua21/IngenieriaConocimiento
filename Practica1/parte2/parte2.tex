\documentclass{uc3mpracticas}
\usepackage{multirow}
\usepackage{listings}



%%%%%%%%%%%%%%%%%%%%%%%%%%%%%%%%%%%%%%%%%%%%%%%%%%%%%%%%%%%%%%%%%%%%%%%%%%%%%%%%
%%%                   Plantilla Prácticas UC3M                               %%%
%%%                Universidad Carlos III de Madrid                          %%%
%%%                   Alejandro Valverde Mahou                               %%%
%%%%%%%%%%%%%%%%%%%%%%%%%%%%%%%%%%%%%%%%%%%%%%%%%%%%%%%%%%%%%%%%%%%%%%%%%%%%%%%%

%Permitir cabeceras y pie de páginas personalizados
\pagestyle{fancy}

%Path por defecto de las imágenes
\graphicspath{ {./images/} }

%Declarar formato de encabezado y pie de página de las páginas del documento
\fancypagestyle{doc}{
  %Cabecera
  \headerpr[1]{Parte 2: \textbf{Adquisición del Conocimiento y Conceptualización}}{}{Ingeniería del Conocimiento}
  %Pie de Página
  \footerpr{}{}{{\thepage} de \pageref{LastPage}}
}

%Declarar formato de encabezado y pie del título e indice
\fancypagestyle{titu}{%
  %Cabecera
  \headerpr{}{}{}
  %Pie de Página
  \footerpr{}{}{}
}


\appto\frontmatter{\pagestyle{titu}}
\appto\mainmatter{\pagestyle{doc}}


\begin{document}

  \mainmatter

  \section{Identificación de conceptos, características, valores y relaciones}

  \subsection{Conceptos, características y valores}


  \begin{center}
    \begin{tabular}{|c|c|c|}
    \hline
    \textbf{Concepto} & \textbf{Características} & \textbf{Valores}\\
    \hline \hline \hline
    \multirow{8}{*}{\textit{Sesión}} & \multirow{2}{*}{Nombre del Juego} & Tres en Raya \\
    & & Twister \\
    \cline{2-3}
    & \multirow{6}{*}{Estado} & Entorno del juego preparado \\
    & & Pesona saludada \\
    & & Juego explicado \\
    & & Juego ejecutando \\
    & & Juego terminado \\
    & & Persona despedida \\
    \hline \hline
    \multirow{8}{*}{\textit{Paciente}} & \multirow{2}{*}{Personalidad} & Distraido \\
    & & Impaciente \\
    \cline{2-3}
    & \multirow{2}{*}{Mano} & Derecha \\
    & & Izquierda \\
    \cline{2-3}
    & \multirow{2}{*}{Pie} & Derecho \\
    & & Izquierdo \\
    \cline{2-3}
    & \multirow{2}{*}{Acción} & Colocar ficha \\
    & & Colocar extremidad \\
    \hline \hline
    \multirow{11}{*}{\textit{Robot}} & \multirow{2}{*}{Color de Ojos} & Rojo \\
    & & Verde \\
    \cline{2-3}
    & \multirow{9}{*}{Acción} & Colocar ficha \\
    & & Elegir extremidad \\
    & & Elegir color \\
    & & Saludar \\
    & & Explicar juego \\
    & & Alentar \\
    & & Corregir \\
    & & Terminar juego \\
    & & Despedirse \\
    \hline \hline
    \multirow{14}{*}{\textit{Juego}} & \multirow{2}{*}{Nombre} & Tres en Raya \\
    & & Twister \\
    \cline{2-3}
    & \multirow{2}{*}{Resultado} & Victoria robot \\
    & & Victoria paciente \\
    \cline{2-3}
    & \multirow{8}{*}{Objeto} & Tablero Tres en Raya \\
    & & Ficha O \\
    & & Ficha X \\
    & & Tablero Twister \\
    & & Casilla Roja \\
    & & Casilla Verde \\
    & & Casilla Azul \\
    & & Casilla Amarilla \\
    \cline{2-3}
    & \multirow{2}{*}{Turno} & Turno robot \\
    & & Turno paciente \\
    \hline
    \end{tabular}
  \end{center}

  \subsection{Relaciones}

  \begin{itemize}
    \item \textit{Sesión} tiene \textit{Juego}
    \item \textit{Sesión} tiene \textit{Paciente}
    \item \textit{Sesión} tiene \textit{Robot}
  \end{itemize}


  \section{El razonamiento del experto (reglas del sistema)}

  Cuando el robot detecta que el entorno del juego está preparado saluda al paciente. Después explica el juego y comienza a ejecutarlo.

  \vspace{2mm}

  Si el juego es el Tres en Raya, primero se le asigna una ficha al paciente y luego la otra al robot, después se comienza a jugar por turnos hasta que uno de los dos consigue hacer las tres en raya.

  \vspace{2mm}

  Si el juego es el Twister, el robot elige qué extremidad y en qué color debe colocar el paciente esta extremidad.

  \vspace{2mm}

  Cuando el juego ha terminado el robot se despide del paciente.


  \section{Identificación de las inferencias}

  \begin{lstlisting}
  SI Estado = Entorno del juego preparado
      ENTONCES Robot.accion = Saludar Y Robot.accion = Explicar juego

  SI Nombre del Juego = Tres en Raya Y Turno = Turno robot
      ENTONCES Robot.accion = Colocar ficha

  SI Nombre del Juego = Tres en Raya Y Turno = Turno paciente
      ENTONCES Paciente.accion = Colocar ficha

  SI Nombre del Juego = Tres en Raya
  Y (Resultado = Victoria robot O Resultado = Victoria paciente)
      ENTONCES Estado = Juego terminado

  SI Estado = Juego terminado
      ENTONCES Robot.accion = Despedirse Y Estado = Persona despedida

  SI Nombre del Juego = Twister Y Turno = Turno robot
      ENTONCES Robot.accion = Elegir extremidad Y Robot.accion = Elegir color

  SI Nombre del Juego = Twister Y Turno = Turno paciente
      ENTONCES Paciente.accion = Colocar extremidad

  \end{lstlisting}















\end{document}
