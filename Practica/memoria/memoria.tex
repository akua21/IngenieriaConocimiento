\documentclass{uc3mpracticas}

\usepackage{helvet}
\renewcommand{\familydefault}{\sfdefault}


%%%%%%%%%%%%%%%%%%%%%%%%%%%%%%%%%%%%%%%%%%%%%%%%%%%%%%%%%%%%%%%%%%%%%%%%%%%%%%%%
%%%                   Plantilla Prácticas UC3M                               %%%
%%%                Universidad Carlos III de Madrid                          %%%
%%%                   Alejandro Valverde Mahou                               %%%
%%%%%%%%%%%%%%%%%%%%%%%%%%%%%%%%%%%%%%%%%%%%%%%%%%%%%%%%%%%%%%%%%%%%%%%%%%%%%%%%

%Permitir cabeceras y pie de páginas personalizados
\pagestyle{fancy}

%Path por defecto de las imágenes
\graphicspath{ {./images/} }

%Declarar formato de encabezado y pie de página de las páginas del documento
\fancypagestyle{doc}{
  %Cabecera
  \headerpr[1]{Sistemas de Producción}{}{Ingeniería del Conocimiento}
  %Pie de Página
  \footerpr{}{\textbf{UC3M}}{{\thepage} de \pageref{LastPage}}
}

%Declarar formato de encabezado y pie del título e indice
\fancypagestyle{titu}{%
  %Cabecera
  \headerpr{}{}{}
  %Pie de Página
  \footerpr{}{}{}
}


\appto\frontmatter{\pagestyle{titu}}
\appto\mainmatter{\pagestyle{doc}}


\begin{document}
  %Comienzo formato título
  \frontmatter


  %Portada 1 (Centrado todo)
  \centeredtitle{Images/LogoUC3M.png}{Grado en Ingeniería Informática}{Curso 2020/2021}{Ingeniería del Conocimiento}{Práctica 1: Sistemas de Producción}

  \vspace{55mm}

  \authors{Alba Reinders Sánchez}{100383444}{Alejandro Valverde Mahou}{100383383}{}{}{Grupo 83}{Leganés}

  \newpage

  %Índice
  \tableofcontents

  \newpage

  %Comienzo formato documento general
  \mainmatter

  \section{Introducción}

En esta primera práctica se aborda la implementación de un \textbf{sistema de producción} (\textit{SP}) en CLIPS que lleve a cabo la ejecución de una sesión de un especialista con un paciente en la que interactúan un humano y un robot, donde este último adopta el papel de especialista.

\vspace{2mm}

En concreto, se crea un \textit{SP} con toda la información respecto a los personajes y el entorno en el que se lleva a cabo la interacción, así como posibles desviaciones durante la sesión en las que el robot debe reaccionar y modificar sus acciones en consecuencia.

\vspace{2mm}

La interacción entre el paciente y el robot se realiza a través del desarrollo de 2 juegos: el \textbf{Twister} y las \textbf{3 en raya}. Además, el sistema está adecuado para tratar con pacientes que puedan presentar 2 personalidades distintas, además del comportamiento base: \textbf{despistado} y \textbf{energético}.

\vspace{2mm}

El documento consiste en el manual técnico con la descripción de la implementación, el manual de usuario con la explicación de cómo usar el programa, las pruebas realizadas y el análisis de los resultados, y para finalizar una serie de conclusiones y comentarios personales.

  \section{Manual técnico}

  \subsection{Ontología}

  \subsection{Reglas}


  \section{Manual de usuario}

  \section{Pruebas realizadas}

  \section{Conclusiones}

  \section{Comentarios personales}




\end{document}
